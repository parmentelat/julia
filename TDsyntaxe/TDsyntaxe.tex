\documentclass{article}

\usepackage{xcolor}
\definecolor{jrouge}{HTML}{CB3C33}
\definecolor{jvert}{HTML}{389826}
\definecolor{jbleu}{HTML}{4063D8}
\definecolor{jviolet}{HTML}{9558B2}
\definecolor{lightred}{HTML}{fcf3f3}
\definecolor{lightgreen}{HTML}{e1f6db}
\definecolor{lightpurple}{HTML}{f4eef7}
\definecolor{lightgrey}{gray}{0.95}

\usepackage{natbib}
%\renewcommand{\citenumfont}[1]{{\tiny#1}}
\renewcommand{\citenumfont}[1]{}
\bibpunct{}{};s;;

\usepackage[top=1.5cm,bottom=1.5cm, left=2.5cm,right=2.5cm]{geometry}

\usepackage{subcaption}

\usepackage{array}
\usepackage{multirow}
\usepackage{setspace}
\usepackage{soul}
\usepackage{amssymb}
\usepackage{mathrsfs}
\usepackage{bbm}
\usepackage{svg}

% unicode-math better loaded after amssymb and other maths packages
% Alternative for XeLaTeX:
\usepackage{ifxetex}

\ifxetex
	% \usefonttheme{professionalfonts}
	\usepackage{fontspec}
	\setmonofont{JuliaMono}
	\usepackage{polyglossia}
	\setmainlanguage{french}
	\usepackage{unicode-math}
\else
	\usepackage[french]{babel}
	\usepackage[mathletters]{ucs}
	\usepackage[utf8x]{inputenc}
	\usepackage[T1]{fontenc}
	% Or whatever. Note that the encoding and the font should match. If T1
	% does not look nice, try deleting the line with the fontenc.
\fi

\usepackage{tikz}
\usetikzlibrary{scopes, backgrounds, arrows, automata, positioning, patterns, calc, decorations.pathmorphing, decorations.pathreplacing, arrows.meta}

\usepackage{hyperref}
\hypersetup{
	colorlinks=true,
	breaklinks=true,
}
\usepackage{ragged2e}
\usepackage{graphicx}
\usepackage{amsmath}
\usepackage{aeguill}

\usepackage{algorithm}
\usepackage{algpseudocode}
\usepackage{stmaryrd}
\usepackage{siunitx}

\usepackage{forloop}
\newcounter{loop}
\newcounter{numEx}
\newcommand{\exo}[1]{
	\stepcounter{numEx}
	\setcounter{loop}{0}
	\subsection*{Exercice \arabic{numEx} -- #1}
}
\newcommand{\correction}{\textsl{\\Correction de l'exercice \arabic{numEx}\\}}
\newenvironment{corr}{
	\correction	\begin{enumerate}}
	{\end{enumerate}
}

\usepackage{enumitem}
\renewlist{itemize}{itemize}{1}
\setlist[itemize]{label={\textbullet}}

\newcommand{\llbra}{\left\llbracket}
\newcommand{\rrbra}{\right\rrbracket}
\renewcommand{\brack}[1]{\ensuremath{\llbra#1\rrbra}}
\newcommand{\der}[2]{#1^{\ensuremath{\left(#2\right)}}}
\newcommand{\paren}[1]{\ensuremath{\left(#1\right)}}
\newcommand{\abs}[1]{\ensuremath{\left|#1\right|}}
\newcommand{\interval}[1]{\ensuremath{\left[#1\right]}}
\newcommand{\set}[2]{\ensuremath{\left\{#1\,\middle|\,#2\right\}}}
\newcommand{\cont}[1]{\mathcal{C}^{#1}}
\newcommand{\tends}[2]{\underset{#1\to #2}{\longrightarrow}}
\newcommand{\seq}[3]{\ensuremath{\left(#1_{#2}\right)_{#2\in#3}}}
\newcommand{\matr}[2]{\mathcal{M}_{#1}\paren{#2}}
\newcommand{\matrRect}[3]{\mathcal{M}_{#1,#2}\paren{#3}}
\newcommand{\Id}{\text{Id}}
\newenvironment{disj}[1]{\left\{\begin{array}{#1}} {\end{array}\right.}
\newcommand{\N}{\mathbb{N}}
\newcommand{\R}{\mathbb{R}}
\newcommand{\C}{\mathbb{C}}
\newcommand{\Q}{\mathbb{Q}}
\newcommand{\Z}{\mathbb{Z}}
\newcommand{\K}{\mathbb{K}}
\newcommand{\eps}{\varepsilon}



\title{TD -- Apprentissage de la programmation en Julia}

\author{Lionel~Zoubritzky}

\date{11/2024}

\usepackage{minted}
\usemintedstyle{paraiso-light}
\setminted[julia]{mathescape,linenos,obeytabs,tabsize=4,numbersep=3pt,fontsize=\small,framesep=2mm,autogobble,bgcolor=lightred,escapeinside=££}
\setminted[bash]{mathescape,obeytabs,tabsize=4,numbersep=3pt,fontsize=\small,framesep=2mm,autogobble,bgcolor=lightgrey,escapeinside=££}

\usepackage{lmodern}
%\newcommand{\jl}[1]{\colorbox{lightred}{\small\ttfamily #1}}
\newmintinline[jl]{julia}{}
\newmintinline[jlscript]{julia}{fontsize=\scriptsize}
\newmint[JL]{julia}{}
\newmint[JLa]{julia}{linenos=false}
\newminted{julia}{}
\newenvironment{julia}{\vspace{-0.6em}\VerbatimEnvironment\begin{juliacode}}{\end{juliacode}}
\newminted[jlrepl]{julia}{linenos=false}
\newenvironment{repl}{\vspace{-0.6em}\VerbatimEnvironment\begin{jlrepl}}{\end{jlrepl}}
\newcommand{\q}{\textquotesingle}
\newcommand{\qq}{\textquotedbl}
\newcommand{\jlREPL}{\textcolor{jvert}{\bfseries julia>}}

\DeclareTextFontCommand{\emph}{\color{jrouge}\bfseries}

\usepackage{xspace}
\newcommand{\expr}{\ensuremath{\left\langle\textit{expr}\right\rangle}\xspace}
\newcommand{\expra}[1]{\ensuremath{\left\langle\textit{expr}_{#1}\right\rangle}\xspace}
\newcommand{\bexpr}{\ensuremath{\left\langle\textit{bexpr}\right\rangle}\xspace}
\newcommand{\bexpra}[1]{\ensuremath{\left\langle\textit{bexpr}_{#1}\right\rangle}\xspace}



\begin{document}
	
\begin{center}
	\Large Apprentissage de la programmation en Julia
	
	TD \no1 : syntaxe et bases du langage
	\vspace{2em}
\end{center}

Les questions et exercices annotés du symbole $(*)$ ne sont pas à traiter en priorité.

À chaque question pour laquelle il faut programmer une fonctionnalité, il est implicitement demandé d'inclure un ou plusieurs tests pour s'assurer que l'implémentation est correcte.

Toute fonction non-auxiliaire doit aussi inclure une docstring. Les commentaires sont partout bienvenus.


\section{Théorie des nombres}

\exo{La suite de Syracuse}

La suite de Syracuse est définie comme la suite d'entiers de premier terme $u_0\in\N^*$ et définie par récurrence avec, pour tout $i\in\N$ :
\begin{itemize}
	\item si $u_i$ est pair alors $u_{i+1} = u_i/2$ ;
	\item sinon, $u_{i+1} = 3u_i + 1$.
\end{itemize}

Écrire une fonction \jl{syracuse(n)} qui renvoie le premier indice $\jl{m}\in\N$ tel que la suite de Syracuse définie par $u_0 = \jl{n}$ vérifie $u_{\jl{m}} = 1$. La conjecture de Syracuse affirme que ce $\jl{m}$ existe toujours.

\exo{Nombres premiers}

Écrire une fonction \jl{isprime(n)} qui renvoie \jl{true} si \jl{n} est premier, et \jl{false} sinon.

\exo{La conjecture de Goldbach}

La conjecture de Goldbach, à ce jour non démontrée, affirme que tout nombre entier pair supérieur à 3 peut s’écrire comme une somme de deux nombres premiers. La conjecture faible de Goldbach affirme, elle, que tout nombre entier impair supérieur à 9 peut s'écrire comme une somme de trois nombres premiers impairs.

Écrire une fonction \jl{goldbach(n)} qui pour tout $\jl{n} \ge 9$ renvoie un triplets d'entiers \jl{(a,b,c)} tel que :
\begin{itemize}
	\item Si \jl{n} est pair, alors \jl{c == 0}, \jl{a} et \jl{b} sont premiers et \jl{n == a + b}.
	\item Sinon, alors \jl{a}, \jl{b} et \jl{c} sont premiers et impairs, et \jl{n == a + b + c}.
\end{itemize}

\exo{La conjecture d'Euler}

\begin{enumerate}
	\item La conjecture d'Euler affirme que pour tout $n>2$, la somme de $n-1$ puissances $n$-ièmes n'est jamais une puissance $n$-ième. En d'autres termes :\vspace{-0.5em}
	\[\forall n>2,\ \forall b\in\N,\ \forall (a_1, \ldots, a_{n-1})\in\paren{\N^*}^{n-1},\ a_1^n + \ldots + a_{n-1}^n \neq b^n\]
	Cette conjecture s'est finalement révélée fausse. Trouver numériquement un contre-exemple pour $n = 5$.
	
	\textsl{On pourra si besoin utiliser la fonction \jl{round(Int, x)} qui renvoie le nombre entier le plus proche du nombre flottant \jl{x}.}

	\item $(*)$ Le problème de trouver des sommes de $n$ (et non pas $n-1$) puissances $n$-ièmes égales à une puissance $n$-ième est plus simple. Écrire une fonction \jl{noteuler(n)} qui renvoie un entier naturel \jl{b} tel qu'il existe un $n$-uplet $(a_1, \ldots, a_n)\in\paren{\N^*}^n$ pour lequel $a_1^n + \ldots + a_n^n = \jl{b}$. La fonction doit aussi afficher les valeurs de $a_1$, \ldots, $a_n$ correspondantes.

	Tester cette fonction pour $n = $ 1, 2, 3 et 5.
\end{enumerate}


\section{Mémoïsation et programmation dynamique}

Dans cette section, on s'intéresse à la mémoïsation, qui est une technique de programmation consistant à retenir les résultats calculés en cours de route, pour ne pas avoir à les recalculer ensuite. Pour illustrer cette idée, on définit une suite de Fibonacci généralisée $\seq un\N$ par récurrence avec $u_1 = a\in\N$, $u_2 = b\in\N$ et $\forall i>2,\ u_i = u_{i-1} + u_{i-2}$. $a$ et $b$ sont des données du problème. La question est de calculer efficacement le terme général d'une telle suite.

Pour ce faire, on définit un tableau \jl{memory} qui contient les termes de la suite de Fibonacci déjà calculés. À chaque appel de \jl{fibonacci(n)}, la fonction que l'on veut produire, le tableau est mis à jour en ajoutant les éléments nécessaires jusqu'à atteindre le terme \jl{n} demandé. Ainsi, le travail effectué lors d'un appel n'a pas besoin d'être reproduit lors d'un appel suivant, il est gardé en mémoire.

La déclaration du tableau \jl{memory} ainsi que la fonction \jl{fib(n)} qui travaille sur le tableau avant de renvoyer la valeur du \jl{n}-ième terme, sont encapsulées dans une fonction \jl{fibonacci_memoization}, qui prend en entrée les paramètres $a$ et $b$, et renvoie \jl{fib}. La fonction \jl{fibonacci} est simplement définie comme la valeur renvoyée par \jl{fibonacci_memoization}, c'est-à-dire la fonction intérieure \jl{fib}.\vspace{0.5em}

\begin{repl}
function fibonacci_memoization(a, b)
    memory = [a, b] # already computed values of Fibonacci's sequence
    function fib(n)
        for i in (length(memory)+1):n # if n <= length(memory), this loop is empty
            push!(memory, memory[end] + memory[end-1])
        end
        memory[n] # return value of fib
    end
    fib
end

fibonacci = fibonacci_memoization(1, 1) # standard Fibonacci sequence
\end{repl}

\exo{Suite de Hofstadter--Conway}

On définit la suite $\seq un\N$ par récurrence avec $u_1 = u_2 = 1$ et $\forall n>2,\ u_n = u_{u_{n-1}} + u_{n - u_{n-1}}$.

Déterminer la limite empirique de $u_n/n$ lorsque $n\to\infty$. À quel point le millionième terme est-il proche de la limite ?

\exo{$(*)$ Nombres premiers mémoïsés}

Réécrire la fonction \jl{isprime} de l'exercice 2 en mémoïsant le caractère premier ou non-premier des nombres rencontrés au fur et à mesure. Quantifier l'accélération résultante pour la fonction \jl{goldbach} de l'exercice 3.

\exo{Problème du sac à dos}

On considère un sac à dos pouvant contenir un poids maximal \jl{wmax} et \jl{N} objets, ayant chacun un poids \jl{weights[i]} et une valeur marchande \jl{costs[i]}, avec \jl{i} variant de 1 à \jl{N}.
Le problème du sac à dos (\textit{knapsack problem}) consiste à choisir quels objets mettre dans le sac pour maximiser la valeur totale sans dépasser le poids maximal.

Pour tester les solutions au problème, on prendre des objets tels que le \jl{i}-ème a pour valeur \jl{3i} et pour poids \jl{i + i÷2 + 2(i%2) - i%3}, pour différentes valeurs de \jl{N} et avec \jl{wmax == N^2÷2}.

\begin{enumerate}
	\item Implémenter les trois algorithmes gloutons\footnote{Un algorithme est dit ``glouton'' lorsqu'il procède en choisissant à chaque étape le choix localement optimal. Un tel algorithme peut être globalement optimal mais ne l'est pas nécessairement.} suivants :
	\begin{enumerate}
		\item Ajouter les objets par ordre décroissant de valeur jusqu'à remplir le sac.
		\item Ajouter les objets par ordre croissant de poids jusqu'à remplir le sac.
		\item Ajouter les objets par ordre décroissant de valeur / poids jusqu'à remplir le sac.
	\end{enumerate}

	Lequel des trois algorithmes donne la meilleure valeur totale ? Est-il optimal ?

	\item On cherche maintenant une solution exacte. Pour cela, remarquons que si l'on dispose de la solution exacte pour les \jl{i - 1} premiers objets pour tout poids maximal inférieur ou égal à \jl{wmax}, alors on peut en déduire la solution exacte pour les \jl{i} premiers objets:
	\begin{itemize}
		\item soit l'objet \jl{i} ne fait pas partie de la solution optimale, et il suffit de garder donc la solution optimale pour les \jl{i - 1} premiers objets et le poids maximal \jl{wmax} ;
		\item soit l'objet \jl{i} fait partie de la solution optimale, donc celle-ci est constituée d'une part du dernier objet et d'autre part de la solution optimale pour les \jl{i - 1} premiers objets et le poids maximal \jl{wmax - weights[i]}.
	\end{itemize}

	Implémenter un algorithme de programmation dynamique\footnote{La programmation dynamique est une méthode algorithmique qui consiste à découper le problème en sous-problèmes et à les résoudre par ordre croissant de taille, en mémoïsant les résultats intermédiaires.} qui exploite ce raisonnement en construisant au fur et à mesure la matrice \jl{opt} dont chaque case \jl{opt[i,w]} contient la liste des objets à garder pour avoir la solution optimale sur les \jl{i} premiers objets avec un poids maximal \jl{w}.
\end{enumerate}

\section{Traitement de texte}

\exo{Parcours de chaîne de caractères}

\begin{enumerate}
	\item On définit la constante suivante : \jl{const REFCHARS = [Char.(33:126); Char.(192:255); 'α':'ω']}. Que fait cette opération ?

	\item On utilise la bibliothèque standard \jl{Random} incluse avec Julia : pour cela, il faut écrire \jl{using Random} dans son code.

	Que fait la fonction \jl{randstring} ? L'utiliser pour générer une chaîne de caractères \jl{str₀} de taille \num{100000000} dont les caractères sont pris dans \jl{REFCHARS}.

	\item Écrire une fonction qui prend en argument une chaîne de caractères et qui identifie la plus longue sous-chaîne ne contenant aucun chiffre. Quelle est sa taille dans \jl{str₀} ?

	\item Écrire une fonction qui calcule le nombre de caractères ASCII dans une chaîne de caractères. On rappelle que les caractères ASCII sont ceux que l'on peut obtenir comme \jl{Char(i)} où $\jl{i}\in\brack{0, 127}$. Quelle est la proportion de caractères ASCII dans \jl{str₀} ? $(*)$ Quelle est la proportion théorique si la taille de \jl{str₀} tend vers $+\infty$ ?

	\item Écrire une fonction qui calcule le nombre d'occurrences où deux lettres consécutives sont identiques. On comptera par exemple deux telles occurrences dans la sous-chaîne ``axxxb"". Quel est le nombre de telles occurrences dans \jl{str₀} ? $(*)$ À quel nombre pouvait-on s'attendre ?

	\item Un palindrome est un mot qui peut se lire identiquement dans les deux sens comme par exemple ``kayak'' ou bien ``engagelejeuquejelegagne''.
	\begin{enumerate}
		\item Écrire une fonction \jl{ispalindrome} qui détermine si une chaîne de caractères est un palindrome. On pourra utiliser la fonction \jl{Iterators.reverse} au besoin.

		\item $(*)$ Écrire une fonction \jl{maxpalindrome} qui extrait le plus long palindrome d'une chaîne de caractères. Quel est le plus long palindrome obtenu sur \jl{str₀} ?
	\end{enumerate}
\end{enumerate}

\exo{$(*)$ Un peu de bio-informatique}

On considère des séquences d'ADN constituées d'un enchaînement de caractères \jl{'A'}, \jl{'T'}, \jl{'G'} et \jl{'C'}. On peut diviser virtuellement une telle séquence en codons, c'est-à-dire en triplets consécutifs. Le chaîne \jl{AAGCTC} est par exemple constituée des deux codons \jl{AAG} et \jl{CTC}.
\begin{enumerate}
	\item Écrire une fonction qui calcule la fréquence d'apparition de chaque codon dans une chaîne de caractères. La valeur renvoyée est un dictionnaire qui associe à chaque codon possible sa fréquence.

	\item Une sous-chaîne \jl{s} est répétée lorsqu'il existe deux sous-chaînes de la séquence \jl{s₁} et \jl{s₂} \underline{disjointes} toutes deux égales à \jl{s}. Écrire une fonction qui identifie la sous-chaîne de \jl{n} codons le plus souvent répété dans une séquence, où \jl{n} est un argument de la fonction. La fonction doit renvoyer la sous-chaîne ainsi que les positions de ses occurrences.

	\item Écrire une fonction qui identifie la sous-chaîne la plus longue qui est répétée dans la séquence (on ne considère que les sous-chaînes constituées de codons, donc de taille multiple de 3).
\end{enumerate}

\end{document}